\documentclass[11pt, oneside]{article}
\usepackage[letterpaper, margin=2cm]{geometry}
\usepackage{MATH517}

\begin{document}
\noindent \textbf{\Large{Caleb Logemann \\
MATH 517 Finite Difference Methods \\
Homework 2
}}

%\lstinputlisting[language=Matlab]{H01_23.m}
\begin{enumerate}
    \item % #1 Done
        Consider the 2-pt boundary value problem:
        \begin{align*}
            -u'' = f(x)\text{ on } 0 < x < L \\
            u(0) = \alpha, \quad u'(L) = \sigma.
        \end{align*}
        Discretize this problem using the $O(h^2)$ central finite differences
        and a ghost point near $x = L$ to handle the Neumann boundary condition.
        Write out the resulting linear system.

        To discretize this problem let $x_i = ih$ where $h = \frac{L}{N + 1}$
        and $N$ is the number of points in the discretization.
        Thus the solution to this BVP will be approximated on these grid points.
        Let $U_i$ be the approximate value of $u(x_i)$.
        Thus an approximate solution to this BVP will be the values of $U_i$ for
        $0 \le i \le N+1$.
        From the boundary conditions, it can be noted that $U_0 = \alpha$.
        The other boundary condition can be handled by introducing a ghost point
        $U_{N+2}$ with the following two conditions
        \begin{align*}
            \frac{1}{h^2}\p{-U_{N} + 2U_{N+1} - U_{N+2}} &= f(x_{N+1}) = f(L)
            \frac{1}{2h} \p{U_{N+2} - U_{N}} &= \sigma. \\
        \end{align*}
        In other words the central finite difference for the second derivative
        must hold on $U_{N}$, $2_{N+1}$, and $U_{N+2}$, and the central
        difference for the first derivative must equal $\sigma$ when centered
        on $U_{N+1}$.
        These two equations can be combined so that the ghost point $U_{N+2}$
        no longer appears in our finite difference method.
        \[
            \frac{1}{h^2}\p{-2U_N + 2U_{N+1}} = f(x_{N+1}) + \frac{2\sigma}{h}
        \]

        Now instead of a system of $N$ equations for $i = 1, 2, \ldots, N$,
        we have a system of $N+1$ equations for $i = 1, 2, \ldots, N+1$.
        For $i = 2, \ldots, N$, the central finite difference can be applied
        directly to the PDE to result in the following $N-1$ equations.
        \[
            \frac{1}{h^2}\p{-U_{i-1} + 2U_{i} - U_{i+1}} = f(x_i)
        \]
        For $i = 1$, we simply replace $U_0$ with $\alpha$ and simplify to
        \[
            \frac{1}{h^2}\p{2U_{1} - U_{2}} = f(x_1) + \frac{\alpha}{h^2}
        \]
        For $i = N + 1$, we use the equation we found before after removing
        the ghost point.
        \[
            \frac{1}{h^2}\p{-2U_N + 2U_{N+1}} = f(x_{N+1}) + \frac{2\sigma}{h}
        \]

        Then this system of equations cen be written as a matrix equation as follows
        \begin{align*}
            A\v{U} &= \v{f} \\
            \intertext{where}
            \v{U} &= \br{U_1, U_2, \cdots, U_{N+1}}^T \\
            \v{f} &= \br{f(x_1) + \frac{\alpha}{h^2}, f(x_2), \cdots, f(x_N), f(x_{N+1}) + \frac{2\sigma}{h}}^T \\
            A &= \frac{1}{h^2}
            \begin{bmatrix}
                 2 & -1     &        &        &    \\
                -1 &  2     & -1     &        &    \\
                   & \ddots & \ddots & \ddots &    \\
                   &        &     -1 &      2 & -1 \\
                   &        &        &     -2 &  2 \\
            \end{bmatrix}
        \end{align*}

    \item % #2
        Find the Green's function that satisfies:
        \[
            -G'' = \delta(x - \xi),\quad G(0;\xi) = 0,\quad G'(L;\xi) = 0.
        \]

    \item % #3
        Use the result from Problem 2 to write out the exact solution to the
        boundary value problem with general $f(x)$, $\alpha$, and $\sigma$.

    \item % #4
        Use the results from Problems 2 and 3 to find the exact inverse to the
        finite difference matrix found in Problem 1.

    \item % #5
        Use the result in Problem 4 to prove that the finite difference method
        in Problem 1 is $L_{\infty}$-stable.

    \item % #6 Done
        Consider the uniform mesh $x_j = jh$ and let
        \[
            U_j = u(x_j) \quad \text{and} \quad W_j \approx u'(x_j).
        \]
        In standard finite differences, we typically find linear combinations
        of $U_j$ to define the approximation $W_i$ to $u'(x_j)$:
        \[
            W_i = \sum*{j}{}{\beta_j U_j}
        \]
        In compact finite differences we are allowed to generalize this to
        \[
            \sum*{j}{}{\alpha_j W_j} = \sum*{j}{}{\beta_j U_j}
        \]

        Find the compact finite difference with the optimal local truncation
        error that has the following form:
        \[
            \alpha W_{j-1} + W_j + \alpha W_{j+1} = \beta\p{\frac{U_{j+1} - U_{j-1}}{2h}}.
        \]

        We can find the local truncation error by inserting the exact solution
        into the finite difference equation and using Taylor series.
        \begin{align*}
            \tau_j &= -\alpha u'(x_{j-1}) - u'(x_j) - \alpha u'(x_{j+1}) + \beta\p{\frac{u(x_{j+1}) - u(x_{j-1})}{2h}} \\
            u(x_{j-1}) &= u(x_j) - h u'(x_j) + \frac{h^2}{2} u''(x_j) - \frac{h^3}{6} u'''(x_j) + \frac{h^4}{24} u^{(4)}(x_j) - \frac{h^5}{120} u^{(5)}(x_j) + O(h^6) \\
            u(x_{j+1}) &= u(x_j) + h u'(x_j) + \frac{h^2}{2} u''(x_j) + \frac{h^3}{6} u'''(x_j) + \frac{h^4}{24} u^{(4)}(x_j) + \frac{h^5}{120} u^{(5)}(x_j) + O(h^6) \\
            u'(x_{j-1}) &= u'(x_j) - h u''(x_j) + \frac{h^2}{2} u'''(x_j) - \frac{h^3}{6} u^{(4)}(x_j) + \frac{h^4}{24} u^{(5)}(x_j) - \frac{h^5}{120} u^{(6)}(x_j) + O(h^6) \\
            u'(x_{j+1}) &= u'(x_j) + h u''(x_j) + \frac{h^2}{2} u'''(x_j) + \frac{h^3}{6} u^{(4)}(x_j) + \frac{h^4}{24} u^{(5)}(x_j) + \frac{h^5}{120} u^{(6)}(x_j) + O(h^6)
            \intertext{Substituting in these Taylor series and simplifying results in}
            \tau_j &= (-1 - 2\alpha + \beta) u'(x_j) + \frac{h^2}{6} (-6 \alpha + \beta) u^{(3)}(x_j) + \frac{h^4}{120} (-10 \alpha + \beta) u^{(5)}(x_j) + O(h^6)
        \end{align*}
        To make the local truncation error as small as possible, we must choose
        $\alpha$ and $\beta$ such that the following two equations are satisfied.
        \begin{align*}
            0 &= -1 - 2\alpha + \beta \\
            0 &= -6 \alpha + \beta
        \end{align*}
        If both of these equations are satisfied then this finite difference
        will be a fourth order approximation.
        These two equations can be solved as follows.
        \begin{align*}
            \beta &= 6 \alpha \\
            0 &= -1 -2 \alpha + 6 \alpha \\
            1 &= 4 \alpha \\
            \alpha &= \frac{1}{4} \\
            \beta &= \frac{3}{2}
        \end{align*}
        If $\alpha$ and $\beta$ are set to these values, then the local
        truncation error becomes
        \begin{align*}
            \tau_j &= \frac{h^4}{120} (-10 \alpha + \beta) u^{(5)}(x_j) + O(h^6)
        \end{align*}.

        Thus the compact finite difference operator will be
        \[
            \frac{1}{4} W_{j-1} + W_j + \frac{1}{4} W_{j+1} = 3\p{\frac{U_{j+1} - U_{j-1}}{4h}}.
        \]
        which is 4th order accurate, because the local truncation error is $\tau = O(h^4)$.

    \item % #7
        Consider Poisson's equation in 2D:
        \begin{align*}
            -u_{xx} - u_{yy} = f(x,y)\text{ in }\Omega = \br{0,1} \times \br{0,1}, \\
            u = g(x,y)\text{ on } \partial\Omega
        \end{align*}
        Discretize this equation using the 5-point Laplacian on a uniform mesh
        $\Delta x = \Delta y = h$.
        Use the standard natural row-wise ordering.

        First we need to specify the discretization of the space $\Omega$.
        Let $x_i = ih$ for $i = 0, 1, \ldots, N+1$ and let $y_j = jh$ for
        $j = 0, 1, \ldots, N+1$, where $h = \frac{1}{N+1}$.
        Then the solution to this PDE can be described by approximating u on
        this mesh, that is $U_{i,j} \approx u(x_i, y_j)$ is an approximation
        to the exact solution.

        Now we can apply finite differences to this PDE.
        The 5-point Laplacian on a uniform mesh is given by 
        \[
            \Delta u \approx \frac{1}{h^2}\p{U_{i-1,j} + U_{i+1,j} + U_{i,j-1} + U_{i,j+1} - 4U_{i,j}}.
        \]
        Using this finite difference in the PDE results in the following set
        of equations
        \[
            \frac{1}{h^2}\p{4U_{i,j} - U_{i-1,j} - U_{i+1,j} - U_{i,j-1} - U_{i,j+1}} = f_{ij}$
        \]
        where $f_{ij} = f(x_i, y_j)$.

        Now in order to turn this into a linear system, a new numbering scheme
        needs to be imposed.
        I will use the natural row-wise ordering where each point is numbered
        along the rows starting at $U_{1,1} = U_1$.
        In general $U_k = U_{i,j}$ when $k = i + (j-1)N$.

        Using this numbering scheme

    \item % #8
        Write a MATLAB code that constructs the sparse coefficient matrix $A$
        and the appropriate right hand side vector $\v{F}$.

    \item % #9
        Using your code, do a numerical convergence study for the following
        right-hand side forcing and exact solution:
        \[
            f(x,y) = -1.25e^{x + .5y}\quad\text{and}\quad u(x,y) = e^{x + .5y}
        \]
        % use the backslash operator in MATLAB
\end{enumerate}
\end{document}
