\documentclass[11pt, oneside]{article}
\usepackage[letterpaper, margin=2cm]{geometry}
\usepackage{MATH517}

\begin{document}
\noindent \textbf{\Large{Caleb Logemann \\
MATH 517 Finite Difference Methods \\
Homework 2
}}

%\lstinputlisting[language=Matlab]{H01_23.m}
\begin{enumerate}
    \item % #1
        Consider the 2-pt boundary value problem:
        \begin{align*}
            -u'' = f(x)\text{ on } 0 < x < L \\
            u(0) = \alpha, \quad u'(L) = \sigma.
        \end{align*}
        Discretize this problem using the $O(h^2)$ central finite differences
        and a ghost point near $x = L$ to handle the Neumann boundary condition.
        Write out the resulting linear system.

        To discretize this problem let $x_i = ih$ where $h = \frac{L}{N + 1}$
        and $N$ is the number of points in the discretization.

    \item % #2
        Find the Green's function that satisfies:
        \[
            -G'' = \delta(x - \xi),\quad G(0;\xi) = 0,\quad G'(L;\xi) = 0.
        \]

    \item % #3
        Use the result from Problem 2 to write out the exact solution to the
        boundary value problem with general $f(x)$, $\alpha$, and $\sigma$.

    \item % #4
        Use the results from Problems 2 and 3 to find the exact inverse to the
        finite difference matrix found in Problem 1.

    \item % #5
        Use the result in Problem 4 to prove that the finite difference method
        in Problem 1 is $L_{\infty}$-stable.

    \item % #6
        Consider the uniform mesh $x_j = jh$ and let
        \[
            U_j = u(x_j) \quad \text{and} \quad W_j \approx u'(x_j).
        \]
        In standard finite differences, we typically find linear combinations
        of $U_j$ to define the approximation $W_i$ to $u'(x_j)$:
        \[
            W_i = \sum*{j}{}{\beta_j U_j}
        \]
        In compact finite differences we are allowed to generalize this to
        \[
            \sum*{j}{}{\alpha_j W_j} = \sum*{j}{}{\beta_j U_j}
        \]

        Find the compact finite difference with the optimal local truncation
        error that has the following form:
        \[
            \alpha W_{j-1} + W_j + \alpha W_{j+1} = \beta\p{\frac{U_{j+1} - U_{j-1}}{2h}}.
        \]

        We can find the local truncation error by inserting the exact solution
        into the finite difference equation and using Taylor series.
        \begin{align*}
            \tau_j &= -\alpha u'(x_{j-1}) - u'(x_j) - \alpha u'(x_{j+1}) + \beta\p{\frac{u(x_{j+1}) - u(x_{j-1})}{2h}} \\
            u(x_{j-1}) &= u(x_j) - h u'(x_j) + \frac{h^2}{2} u''(x_j) - \frac{h^3}{6} u'''(x_j) + \frac{h^4}{24} u^{(4)}(x_j) - \frac{h^5}{120} u^{(5)}(x_j) + O(h^6) \\
            u(x_{j+1}) &= u(x_j) + h u'(x_j) + \frac{h^2}{2} u''(x_j) + \frac{h^3}{6} u'''(x_j) + \frac{h^4}{24} u^{(4)}(x_j) + \frac{h^5}{120} u^{(5)}(x_j) + O(h^6) \\
            u'(x_{j-1}) &= u'(x_j) - h u''(x_j) + \frac{h^2}{2} u'''(x_j) - \frac{h^3}{6} u^{(4)}(x_j) + \frac{h^4}{24} u^{(5)}(x_j) - \frac{h^5}{120} u^{(6)}(x_j) + O(h^6) \\
            u'(x_{j+1}) &= u'(x_j) + h u''(x_j) + \frac{h^2}{2} u'''(x_j) + \frac{h^3}{6} u^{(4)}(x_j) + \frac{h^4}{24} u^{(5)}(x_j) + \frac{h^5}{120} u^{(6)}(x_j) + O(h^6)
            \intertext{Substituting in these Taylor series and simplifying results in}
            \tau_j &= (-1 - 2\alpha + \beta) u'(x_j) + \frac{h^2}{6} (-6 \alpha + \beta) u^{(3)}(x_j) + \frac{h^4}{120} (-10 \alpha + \beta) u^{(5)}(x_j)
        \end{align*}
        To make the local truncation error as small as possible, we must choose
        $\alpha$ and $\beta$ such that the following two equations are satisfied.
        \begin{align*}
            0 &= -1 - 2\alpha + \beta \\
            0 &= -6 \alpha + \beta
        \end{align*}
        If both of these equations are satisfied then this finite difference
        will be a fourth order approximation.

        \begin{align*}
            
        \end{align*}


    \item % #7
        Consider Poisson's equation in 2D:
        \begin{align*}
            -u_{xx} - u_{yy} = f(x,y)\text{ in }\Omega = \br{0,1} \times \br{0,1}, \\
            u = g(x,y)\text{ on } \partial\Omega
        \end{align*}
        Discretize this equation using the 5-point Laplacian on a uniform mesh
        $\Delta x = \Delta y = h$.
        Use the standard natural row-wise ordering.

    \item % #8
        Write a MATLAB code that constructs the sparse coefficient matrix $A$
        and the appropriate right hand side vector $\v{F}$.

    \item % #9
        Using your code, do a numerical convergence study for the following
        right-hand side forcing and exact solution:
        \[
            f(x,y) = -1.25e^{x + .5y}\quad\text{and}\quad u(x,y) = e^{x + .5y}
        \]
        % use the backslash operator in MATLAB
\end{enumerate}
\end{document}
