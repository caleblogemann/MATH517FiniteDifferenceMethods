\documentclass[11pt, oneside, titlepage]{article}
\usepackage[letterpaper, margin=2cm]{geometry}
\usepackage{MATH517}
\usepackage{Calculus}

\title{MATH 517 Finite Differences Homework 4}
\author{Caleb Logemann}

\begin{document}
\maketitle

%\lstinputlisting[language=Matlab]{H01_23.m}
\begin{enumerate}
    \item % #1
        Consider the Leapfrog method
        \[
            U^{n+1} = U^{n−1} + 2k f(U^n)
        \]
        applied to the test problem $u′ = \lambda u$.
        The method is zero-stable and second order accurate, and hence convergent.
        If $\lambda < 0$ then the true solution is exponentially decaying.

        On the other hand, for $\lambda < 0$ and $k > 0$ the point $z = k\lambda$ is never in the
        region of absolute stability of this method,
        and hence the numerical solution should be growing exponentially for any
        nonzero time step.
        (And yet it converges to a function that is exponentially decaying.)

        Suppose we take $U^0 = \eta$, use Forward Euler to generate $U^1$, and
        then use the midpoint method for $n = 2, 3, \cdots$.
        Work out the exact solution $U^n$ by solving the linear difference
        equation and explain how the apparent paradox described above is resolved

        We know that $U^0 = \eta$ and
        $U^1 = U^0 + k \lambda U^0 = (1 + k\lambda)\eta$ by the Euler method.
        We also know that the solution is governed by the difference equation
        \[
            U^{n+1} - 2k\lambda U^n - U^{n-1} = 0
        \]
        We know that solutions to this difference equation are solutions of the
        following equation
        \[
            \zeta^2 - 2k\lambda \zeta - 1 = 0
        \]

    \item % #2 Done
        \begin{enumerate}
            \item[(a)]
                Find the general solution of the linear difference equation
                \[
                    U^{n+3} + 2U^{n+2} - 4U^{n + 1} - 8U^n = 0
                \]

                The general solution of the linear difference equation can be
                found by finding the roots of the following polynomial.
                \[
                    \zeta^3 + 2\zeta^2 - 4\zeta - 8
                \]
                This polynomial factors to
                \[
                    (\zeta - 2)(\zeta + 2)^2
                \]
                Thus the general solution to this linear difference equation is
                \[
                    U^n = c_1 2^n + c_2 (-2)^n + c_3 n (-2)^n
                \]

            \item[(b)]
                Determine the particular solution with initial data $U_0 = 4$,
                $U_1 = -2$, and $U_2 = 8$.

                The particular solution can be found as follows.
                \begin{align*}
                    U^0 &= 4 = c_1 + c_2 \\
                    c_1 &= 4 - c_2 \\
                    U^1 &= -2 = 2 c_1 - 2 c_2 - 2c_3 \\
                    -2 &= 2(4 - c_2) - 2 c_2 - 2 c_3 \\
                    -10 + 4 c_2 &= -2 c_3 \\
                    c_3 &= 5 - 2 c_2
                    U^2 &= 8 = 4c_1 + 4c_2 + 8c_3 \\
                    8 &= 4(4 - c_2) + 4c_2 + 8(5 - 2c_2) \\
                    8 &= 16 - 4c_2 + 4c_2 + 40 - 16c_2 \\
                    -48 &= -16c_2 \\
                    c_2 &= 3 \\
                    c_1 &= 1 \\
                    c_3 &= -1 
                \end{align*}
                Thus the particular solution is $U^n = 2^n + (3 - n)(-2)^n$.

            \item[(c)]
                Consider the iteration:
                \[
                    \begin{bmatrix}
                        U^{n+1} \\
                        U^{n+2} \\
                        U^{n+3}
                    \end{bmatrix}
                    =
                    \begin{bmatrix}
                        0 & 1 & 0 \\
                        0 & 0 & 1 \\
                        8 & 4 & 2
                    \end{bmatrix}
                    \begin{bmatrix}
                        U^n \\
                        U^{n+1} \\
                        U^{n+2}
                    \end{bmatrix}
                \]
                The matrix appearing here is the companion matrix for the
                difference equation.
                If this matrix is called $A$, then we can determine $U^n$ from
                the starting values if we know $A^n$, the nth power of $A$.
                If $A = R\Lambda R^{-1}$ is the Jordan Canonical form for the
                matrix, then $A^n = R\Lambda^n R^{-1}$.
                Determine the eigenvalues and Jordan canonical form for this
                matrix and show how this is related to the general solution
                found in (a).
                The eigenvalues of $A$ can be found by solving the following

                equation
                \begin{align*}
                    \det(A - \lambda I) &= 0
                    -\lambda^3 - 2\lambda^2 + 4\lambda + 8 &= 0
                \end{align*}
                This equation has the same solutions as the root of the
                polynomial in part (a).
                Therefore the eigenvalues are $\lamba = 2, -2, -2$.
                The Jordan Cononical form of $A$ is
                \[
                    \Lambda =
                    \begin{bmatrix}
                        2 & 0 & 0 \\
                        0 & -2 & 1 \\
                        0 & 0 & -2
                    \end{bmatrix}
                \]
                The eigenvalues correspond directly roots of the characteristic
                polynomial of the lienar difference equation.
        \end{enumerate}

    \item % #3
        Write a MATLAB script to plot the region of absolute stability of the
        4-stage Runge-Kutta method. (See example 5.13 on Page 126)

    \item % #4

\end{enumerate}
\end{document}
