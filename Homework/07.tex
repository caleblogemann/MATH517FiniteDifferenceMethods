\documentclass[11pt, oneside, titlepage]{article}
\usepackage[letterpaper, margin=2cm]{geometry}
\usepackage{MATH517}
\usepackage{Calculus}

\title{MATH 517 Finite Differences Homework 7}
\author{Caleb Logemann}

\begin{document}
\maketitle

%\lstinputlisting[language=Matlab]{H01_23.m}
\begin{enumerate}
    \item % #1
        \begin{enumerate}
            \item[(a)]
                Implement the alternating direction implicit (ADI) scheme for
                this problem.
                Use the backslash operator in MATLAB to invert the necessary
                matrices.

            \item[(b)]
                Do a convergence study of your method for the following exact
                solution:
                \[
                    u(t, x, y) = e^{-32 \pi^2 t} \cos{4 \pi x} \cos{4\pi y}
                \]

            \item[(c)]
                For the problem in (b), put a tic command immediately before you
                solve the first tridiagonal system, and a toc command
                immediately after the second tridiagonal solve.
                Create a table of run times for various $N$ in a single timestep 
                of your solver.
                Comment on your results.
        \end{enumerate}

    \item % #2
        Third Order Lax-Wendroff
        \begin{enumerate}
            \item[(a)]
                Construct a third order accurate Lax-Wendroff-type method for
                the advection equation.

                First I will expand $u(t + k, x)$ using a Taylor series.
                \[
                    u(t + k, x) = u(t, x) + k u_t(t, x) + \frac{k^2}{2}
                        u_{tt}(t, x) + \frac{k^3}{6} u_{ttt}(t, x) + O(k^4)
                \]
                The advection equation states that $u_t = -a u_x$, therefore it
                can also be states that $u_{tt} = a^2 u_{xx}$ and
                $u_{ttt} = -a^3 u_{xxx}$.
                Thus the Taylor expansion for the advection equation becomes
                \[
                    u(t+k,x) = u(t, x)  - ak u_x + \frac{(ak)^2}{2} u_{xx}
                        - \frac{(ak)^3}{6} u_{xxx} + O(k^4)
                \]

                In order to approximate the spacial derivatives, I will create
                a cubic polynomial that interpolates $U_{j-2}^n$, $U_{j-1}^n$,
                $U_j^n$ and $U_{j+1}^n$.
                I will express this polynomial in the form
                \[
                    p(x) = a(x - x_j)^3 + b(x - x_j)^2 + c(x - x_j) + d.
                \]
                This form will make finding the derivatives, $u_x(t^n, x_i)$,
                $u_{xx}(t^n, x_i)$, and $u_{xxx}(t^n, x_i)$, easier in the
                future.
                In order to find the coefficients $a$, $b$, $c$, and $d$, the
                following four equations must be solved.
                \begin{align*}
                    p(x_j - 2h) &= a(-2h)^3 + b(-2h)^2 + c(-2h) + d = U_{j-2}^n
                    p(x_j - h) &= a(-h)^3 + b(-h)^2 + c(-h) + d = U_{j-1}^n
                    p(x_j) &= d = U_{j}^n
                    p(x_j + h) &= a(h)^3 + b(h)^2 + c(h) + d = U_{j+1}^n
                \end{align*}
                After solving these equations in Mathematica, I found the
                coefficients to be
                \begin{align*}
                    a &= -\frac{U_{j-2}^n - 3U_{j-1}^n + 3U_j^n - U_{j+1}^n}{6h^3}\\
                    b &= \frac{U_{j-1}^n - 2U_j^n + U_{j+1}^n}{2h^2}\\
                    c &= \frac{U_{j-2}^n - 6U_{j-1}^n + 3U_j^n + 2U_{j+1}^n}{6h}\\
                    d &= U_j^n \\
                \end{align*}
                Now note that $u(t^n, x) \approx p(x)$ when $x \in \br{x-2h, x+h}$.
                Therefore
                \begin{align*}
                    u_x(t^n, x_j) \approx p'(x_j) = c \\
                    u_{xx}(t^n, x_j) \approx p''(x_j) = 2b \\
                    u_{xxx}(t^n, x_j) \approx p'''(x_j) = 6a
                \end{align*}





                In other words, a cubic polynomial that interpolates $(x_{j-2}, u(t^n, x_{j-2}))$
                following


            \item[(b)]
        \end{enumerate}
    \item % #3
    \item % #4
\end{enumerate}
\end{document}
