\documentclass[11pt, oneside, titlepage]{article}
\usepackage[letterpaper, margin=2cm]{geometry}
\usepackage{MATH517}
\usepackage{Calculus}
\usepackage{booktabs}


\title{MATH 517 Finite Differences Homework 7}
\author{Caleb Logemann}

\begin{document}
\maketitle

%\lstinputlisting[language=Matlab]{H01_23.m}
\begin{enumerate}
    \item % #1 Done
        \begin{enumerate}
            \item[(a)]
                Implement the alternating direction implicit (ADI) scheme for
                this problem.
                Use the backslash operator in MATLAB to invert the necessary
                matrices.

                \lstinputlisting[language=Matlab]{ADI.m}

            \item[(b)]
                Do a convergence study of your method for the following exact
                solution:
                \[
                    u(t, x, y) = e^{-32 \pi^2 t} \cos{4 \pi x} \cos{4\pi y}
                \]

                \lstinputlisting[language=Matlab, lastline=18]{H07.m}

                \begin{verbatim}
                    ans = 

                        hRatios    errorRatios    order 
                        _______    ___________    ______

                        2           8.132         3.0236
                        2          9.4968         3.2474
                        2          12.937         3.6935
                        2          15.027         3.9095
                \end{verbatim}


            \item[(c)]
                For the problem in (b), put a tic command immediately before you
                solve the first tridiagonal system, and a toc command
                immediately after the second tridiagonal solve.
                Create a table of run times for various $N$ in a single timestep 
                of your solver.
                Comment on your results.

                \begin{center}
                    \begin{tabular}{cc}
                        \toprule
                        N & time (sec) \\
                        \midrule
                        19 & 0.002496 \\
                        39 & 0.002069 \\
                        79 & 0.005204 \\
                        159 & 0.011467 \\
                        319 & 0.054771 \\
                        639 & 0.221379 \\
                        \bottomrule
                    \end{tabular}
                \end{center}
                The time for one iteration seems to be growing with $N^2$, or at
                the very least it is growing faster than $N$ is growing.
                When $N$ doubles the time for one iteration increases by more than
                2 times probably closer to 4 times.
                I believe that the time is growing by $N^2$ because the size of
                the system being solved is $N^2$, because we are in 2 dimensions.
                This seems to indicate that this method will not scale well to large
                problems.
        \end{enumerate}

    \item % #2
        Third Order Lax-Wendroff
        \begin{enumerate}
            \item[(a)]
                Construct a third order accurate Lax-Wendroff-type method for
                the advection equation.

                First I will expand $u(t + k, x)$ using a Taylor series.
                \[
                    u(t + k, x) = u(t, x) + k u_t(t, x) + \frac{k^2}{2}
                        u_{tt}(t, x) + \frac{k^3}{6} u_{ttt}(t, x) + O(k^4)
                \]
                The advection equation states that $u_t = -a u_x$, therefore it
                can also be states that $u_{tt} = a^2 u_{xx}$ and
                $u_{ttt} = -a^3 u_{xxx}$.
                Thus the Taylor expansion for the advection equation becomes
                \[
                    u(t+k,x) = u(t, x)  - ak u_x + \frac{(ak)^2}{2} u_{xx}
                        - \frac{(ak)^3}{6} u_{xxx} + O(k^4)
                \]

                In order to approximate the spacial derivatives, I will create
                a cubic polynomial that interpolates $U_{j-2}^n$, $U_{j-1}^n$,
                $U_j^n$ and $U_{j+1}^n$.
                I will express this polynomial in the form
                \[
                    p(x) = a(x - x_j)^3 + b(x - x_j)^2 + c(x - x_j) + d.
                \]
                This form will make finding the derivatives, $u_x(t^n, x_i)$,
                $u_{xx}(t^n, x_i)$, and $u_{xxx}(t^n, x_i)$, easier in the
                future.
                In order to find the coefficients $a$, $b$, $c$, and $d$, the
                following four equations must be solved.
                \begin{align*}
                    p(x_j - 2h) &= a(-2h)^3 + b(-2h)^2 + c(-2h) + d = U_{j-2}^n \\
                    p(x_j - h) &= a(-h)^3 + b(-h)^2 + c(-h) + d = U_{j-1}^n \\
                    p(x_j) &= d = U_{j}^n \\
                    p(x_j + h) &= a(h)^3 + b(h)^2 + c(h) + d = U_{j+1}^n \\
                \end{align*}
                After solving these equations in Mathematica, I found the
                coefficients to be
                \begin{align*}
                    a &= -\frac{U_{j-2}^n - 3U_{j-1}^n + 3U_j^n - U_{j+1}^n}{6h^3}\\
                    b &= \frac{U_{j-1}^n - 2U_j^n + U_{j+1}^n}{2h^2}\\
                    c &= \frac{U_{j-2}^n - 6U_{j-1}^n + 3U_j^n + 2U_{j+1}^n}{6h} \\
                    d &= U_j^n
                \end{align*}
                Now note that $u(t^n, x) \approx p(x)$ when $x \in \br{x-2h, x+h}$.
                Therefore
                \begin{align*}
                    u_x(t^n, x_j) \approx p'(x_j) = c \\
                    u_{xx}(t^n, x_j) \approx p''(x_j) = 2b \\
                    u_{xxx}(t^n, x_j) \approx p'''(x_j) = 6a
                \end{align*}

                Now substituting into the Taylor expansion to actually create a
                numerical method we get
                \begin{align*}
                    U_j^{n+1} &= U_j^n - \frac{ak}{6h} \p{U_{j-2}^n - 6U_{j-1}^n + 3U_j^n + 2U_{j+1}^n} \\
                        &+ \frac{(ak)^2}{2h^2} \p{U_{j-1}^n - 2U_j^n + U_{j+1}^n} \\
                        &+ \frac{(ak)^3}{6h^3} \p{U_{j-2}^n - 3U_{j-1}^n + 3U_j^n - U_{j+1}^n}
                \end{align*}

            \item[(b)]
                Verify that the truncation error is $O(k^3)$ if $h = O(k)$.
        \end{enumerate}

    \item % #3
        Third-Order Method of Lines with RK3: \\
        The following method is a third order accurate Runge-Kutta method for
        $u' = f(u):$
        \begin{align*}
            U^{n+1} = U^n + \frac{k}{9}\p{2Y_1 + 3Y_2 + 4Y_3} \\
            Y_1 = f\p{U^n}, \quad
            Y_2 = f\p{U^n + \frac{k}{2} Y_1}, \quad
            Y_3 = f\p{U^n + \frac{3k}{4} Y_2}
        \end{align*}
        \begin{enumerate}
            \item[(a)]
                Construct a third order accurate method for the advection equation.

                In order to approximate the first derivative, $u_x$ I will use
                the interpolation polynomial that I found in problem 2:
                \[
                    p(x) = a(x - x_j)^3 + b(x - x_j)^2 + c(x - x_j) + d.
                \]
                where
                \begin{align*}
                    a &= -\frac{U_{j-2}^n - 3U_{j-1}^n + 3U_j^n - U_{j+1}^n}{6h^3}\\
                    b &= \frac{U_{j-1}^n - 2U_j^n + U_{j+1}^n}{2h^2}\\
                    c &= \frac{U_{j-2}^n - 6U_{j-1}^n + 3U_j^n + 2U_{j+1}^n}{6h} \\
                    d &= U_j^n.
                \end{align*}
                Originally the advection equation states, on the spacially
                discretized system, that
                \[
                    U_j'(t) = a \frac{\mathrm{d}}{\mathrm{d}x} U_j(t)
                \]
                However we can now replace $\frac{\mathrm{d}}{\mathrm{d}x} U_j(t)$
                with $\frac{\mathrm{d}}{\mathrm{d}x} p(x)$ at $x = x_j$, which is
                \[
                    \frac{U_{j-2}^n - 6U_{j-1}^n + 3U_j^n + 2U_{j+1}^n}{6h}.
                \]
                Now for each $j$, there is an ODE of the form
                \[
                    U_j'(t) = \frac{a}{6h} \p{U_{j-2}^n - 6U_{j-1}^n + 3U_j^n + 2U_{j+1}^n}
                \]
                Now the RK3 method can be applied where
                \[
                    f(U^n) = \frac{a}{6h} \p{U_{j-2}^n - 6U_{j-1}^n + 3U_j^n + 2U_{j+1}^n}
                \]


            \item[(b)]
                Verify that the truncation error is $O(k^3)$ if $k = O(h)$

            \item[(c)]

        \end{enumerate}

    \item % #4

\end{enumerate}
\end{document}
