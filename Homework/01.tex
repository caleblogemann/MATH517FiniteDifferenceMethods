\documentclass[11pt, oneside]{article}
\usepackage[letterpaper, margin=2cm]{geometry}
\usepackage{Logemann}
\usepackage{Vector}

\begin{document}
\noindent \textbf{\Large{Caleb Logemann \\
MATH 517 Finite Difference Methods \\
Homework 1
}}

\begin{enumerate}
    \item % #1 
        Consider a nonuniform grid $x_1 < x_2 < x_3 < x_4$.
        Derive a finite difference approximation of $u''(x_2)$ that is as
        accurate as possible for smooth functions $u(x)$, based on the four
        values $U_1 = u(x_1)$, $U_2 = u(x_2)$, $U_3 = u(x_3)$, and
        $U_4 = u(x_4)$.
        Give an expression for the dominant term in the error.

        First let $h_1 = x_2 - x_1$, $h_2 = x_3 - x_2$ and $h_3 = x_4 - x_3$.
        In order to approximate $u''(x_2)$, we will use a linear combination of
        $U_1, \ldots, U_4$, that is we will find coefficients
        $\omega_1, \ldots, \omega_4$ such that
        $\omega_1 U_1 + \omega_2 U_2 + \omega_3 U_3 + \omega_4 U_4 = u''(x_2) + E$,
        where the error, $E$, is as small as possible.

        $U_1, U_3,$ and $U_4$ can be expressed as Taylor expansions about $U_2$
        as follows
        \begin{align*}
            U_1 &= u(x_1) = u(x_2) + u'(x_2)\p{-h_1} + \frac{1}{2} u''(x_2)
                \p{-h_1}^2 + \frac{1}{6} u'''(x_2) \p{-h_1}^3 +
                \frac{1}{24} u^{(4)}(c_1) \p{-h_1}^4 \\
            U_3 &= u(x_1) = u(x_2) + u'(x_2)\p{h_2} + \frac{1}{2} u''(x_2)
                \p{h_2}^2 + \frac{1}{6} u'''(x_2) \p{h_2}^3 +
                \frac{1}{24} u^{(4)}(c_2) \p{h_2}^4 \\
            U_4 &= u(x_1) = u(x_2) + u'(x_2)\p{h_2 + h_3} + \frac{1}{2} u''(x_2)
                \p{h_2 + h_3}^2 + \frac{1}{6} u'''(x_2) \p{h_2 + h_3}^3 +
                \frac{1}{24} u^{(4)}(c_3) \p{h_2 + h_3}^4
        \end{align*}
        where $c_1 \in \br{x_1, x_2}$, $c_2 \in \br{x_2, x_3}$, and
        $c_3 \in \br{x_2, x_4}$.

        Substituting these Taylor expansions into the linear combination and
        gathering the function and derivative values of $u$ at $x_2$ results in
        \begin{align*}
            \omega_1 U_1 + \omega_2 U_2 + \omega_3 U_3 + \omega_4 U_4 &=
                \p{\omega_1 + \omega_2 + \omega_3 + \omega_4} u(x_2) \\
                &+ \p{-h_1 \omega_1 + h_2 \omega_3 + \p{h_2 + h_3} \omega_4} u'(x_2) \\
                &+ \frac{1}{2}\p{h_1^2 \omega_1 + h_2^2 \omega_3 + \p{h_2 + h_3}^2 \omega_4} u''(x_2) \\
                &+ \frac{1}{6}\p{-h_1^3 \omega_1 + h_2^3 \omega_3 + \p{h_2 + h_3}^3 \omega_4} u'''(x_2) \\
                &+ \frac{1}{24}\p{h_1^4 \omega_1 u^{(4)}(c_1) + h_2^4 \omega_3 u^{(4)}(c_2)+ \p{h_2 + h_3}^4 \omega_4 u^{(4)}(c_3)}
        \end{align*}

        Since there are four coefficients to set in the linear combination we can
        specify up to 4 conditions on these coefficients to get the best possible
        approximation of $u''(x_2)$.
        These equations are as follows
        \begin{align*}
            \omega_1 + \omega_2 + \omega_3 + \omega_4 &= 0 \\
            -h_1 \omega_1 + h_2 \omega_3 + \p{h_2 + h_3} \omega_4 &= 0 \\
            h_1^2 \omega_1 + h_2^2 \omega_3 + \p{h_2 + h_3}^2 \omega_4 &= 2 \\
            -h_1^3 \omega_1 + h_2^3 \omega_3 + \p{h_2 + h_3}^3 \omega_4 &= 0
        \end{align*}
        If these equations are satisfied, then
        \begin{align*}
            \omega_1 U_1 + \omega_2 U_2 + \omega_3 U_3 + \omega_4 U_4 &= u''(x_2) + \frac{1}{24}\p{h_1^4 \omega_1 u^{(4)}(c_1) + h_2^4 \omega_3 u^{(4)}(c_2)+ \p{h_2 + h_3}^4 \omega_4 u^{(4)}(c_3)}
        \end{align*}
        where the approximation is $\omega_1 U_1 + \omega_2 U_2 + \omega_3 U_3 + \omega_4 U_4$
        and the error is $\frac{1}{24}\p{h_1^4 \omega_1 u^{(4)}(c_1) + h_2^4 \omega_3 u^{(4)}(c_2)+ \p{h_2 + h_3}^4 \omega_4 u^{(4)}(c_3)}$

        Using Mathematica, this system of equations can be solved, to find that
        the coefficients are
        \begin{align*}
            \omega_1 &= \frac{2\p{2 h_2 + h_3}}{h_1 \p{h_1 + h_2}\p{h_1 + h_2 + h_3}} \\
            \omega_2 &= \frac{2h_1 - 4 h_2 - 2 h_3}{h_1 h_2^2 + h_1 h_2 h_3} \\
            \omega_3 &= \frac{2\p{-h_1 + h_2 + h_3}}{h_2\p{h_1 + h_2} h_3} \\
            \omega_4 &= \frac{2\p{h_1 - h_2}}{h_3\p{h_2 + h_3}\p{h_1 + h_2 + h_3}}
        \end{align*}

        Since $u$ is a smooth function, the error can be simplified using the
        Intermediate Value Theorem, by noting that
        \begin{align*}
            \frac{h_1^4 \omega_1 u^{(4)}(c_1) + h_2^4 \omega_3 u^{(4)}(c_2)}{h_1^4 \omega_1 + h_2^4 \omega_3} &= u^{(4)}(\rho) \\
            h_1^4 \omega_1 u^{(4)}(c_1) + h_2^4 \omega_3 u^{(4)}(c_2) &= \p{h_1^4 \omega_1 + h_2^4 \omega_3} u^{(4)}(\rho)
        \end{align*}
        for some $\rho \in \br{x_1, x_3}$.
        Thus the error becomes
        \begin{align*}
            \frac{1}{24}\p{\p{h_1^4 \omega_1 + h_2^4 \omega_3}u^{(4)}(\rho)+ \p{h_2 + h_3}^4 \omega_4 u^{(4)}(c_3)}.
        \end{align*}
        The Intermediate Value Theorem can be used again to see that 
        \begin{align*}
            \frac{\p{h_1^4 \omega_1 + h_2^4 \omega_3}u^{(4)}(\rho)+ \p{h_2 + h_3}^4 \omega_4 u^{(4)}(c_3)}{h_1^4 \omega_1 + h_2^4 \omega_3 + \p{h_2 + h_3}^4 \omega_4} &= u^{(4)}(\mu) \\
            \p{h_1^4 \omega_1 + h_2^4 \omega_3}u^{(4)}(\rho)+ \p{h_2 + h_3}^4 \omega_4 u^{(4)}(c_3) &= \p{h_1^4 \omega_1 + h_2^4 \omega_3 + \p{h_2 + h_3}^4 \omega_4} u^{(4)}(\mu)
        \end{align*}
        for $\mu \in \br{x_1, x_4}$.

        The error can thus be written as
        \begin{align*}
            E &= \frac{1}{24}\p{h_1^4 \omega_1 + h_2^4 \omega_3 + \p{h_2 + h_3}^4 \omega_4} u^{(4)}(\mu).
        \end{align*}
        Substituting in for $\omega_1$, $\omega_3$, and $\omega_4$ and
        simplifying results in
        \begin{align*}
            E &= -\frac{1}{12} \p{h_2\p{h_2 + h_3} - h_1 \p{2 h_2 + h_3}} u^{(4)}(\mu).
        \end{align*}

    \item % #2
    \item % #3

    \item % #4
        Consider the following 2-pt BVP: 
        \begin{align*}
            u'' + u = f(x), \quad \text{on} \quad 0 \le x \le 10 \\
            u'(0) - u(0) = 0, \quad u'(10) + u(10) = 0
        \end{align*}
        Construct a second order accurate finite-difference method for this BVP.
        Write your method as a linear system in the form $A \v{u} = \v{f}$.

        First the interval $\br{0, 10}$ needs to be discretized.
        Let $x_0 = 0$, $x_{N + 1} = 10$, and let $x_{i} = ih$ for $1 \le i \le N$, 
        where $h = \frac{10}{N+1}$.
        Thus the interval $\br{0, 10}$ is described as a grid of $N+2$ equally
        spaced points with grid spacing $h$.
        Thus the solution to this BVP will be approximated on these grid points.
        Let $U_i$ be the approximate value of $u(x_i)$.
        Thus an approximate solution to this BVP will be the values of $U_i$ for
        $0 \le i \le N+1$.

        Based on the initial conditions be can find expressions for $U_0$ and
        $U_{N+1}$.

        In order that the finite-difference method is second order accurate, I
        will use the second order central difference to approximate the second
        derivative.
        This finite difference is 
        \begin{align*}
            u''(x_i) \approx \frac{1}{h^2}\p{U_{i-1} - 2U_{i} + U_{i+1}}
        \end{align*}
        for $1 \le i \le N$.

    \item % #5
    \item % #6
    \item % #7
    \item % #8
    \item % #9
    \item % #10
\end{enumerate}

\end{document}
